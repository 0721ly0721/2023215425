%!TeX program = xelatex
% 写自己的报告,建议直接从main.tex基础之上修改,如何使用参考guide.tex。
\documentclass[12pt,hyperref,a4paper,twoside,UTF8]{ctexart}
\usepackage{HFUTReport}
\usepackage{listings}
\usepackage{xcolor}

\usepackage{setspace}
\setstretch{1.5} % 设置全局行距为1.5倍

\usepackage{enumitem} % 载入enumitem包以便自定义列表环境
\setlist[itemize]{itemsep=0pt, parsep=0pt} % 设置itemize环境的项目间距和段落间距

\setmainfont{Times New Roman} % 英文正文为Times New Roman

%封面页设置
{   
    %标题
    \title{ 
        \vspace{1cm}
        \heiti \Huge \textbf{《嵌入式系统及应用》课程} \par
        \vspace{1cm} 
        \heiti \Large {\underline{综合实验报告}}    
        \vspace{3cm}
    }

    \author{
        \vspace{0.5cm}
        \kaishu\Large 学院\ \dlmu[9cm]{仪器科学与光电工程学院} \\ %学院
        \vspace{0.5cm}
        \kaishu\Large 专业\ \dlmu[9cm]{光电信息工程} \\ %班级
        \vspace{0.5cm}
        \kaishu\Large 学号\ \dlmu[9cm]{2024XXXXXX} \qquad  \\ %学号
        \vspace{0.5cm}
        \kaishu\Large 姓名\ \dlmu[9cm]{XXX} \qquad \\ %姓名 
    }
        
    \date{\today} % 默认为今天的日期,可以注释掉不显示日期
}
%%------------------------document环境开始------------------------%%
\begin{document}

%%-----------------------封面--------------------%%
\cover
\thispagestyle{empty} % 首页不显示页码
%%------------------摘要-------------%%
\newpage
\begin{abstract}

本实验通过理论与实践结合,使学生掌握STM32F103单片机的综合开发技能,理解嵌入式系统的核心设计理念,并为后续复杂项目(如物联网终端、工业控制器)的开发奠定坚实基础。

\end{abstract}

\thispagestyle{empty} % 首页不显示页码

%%--------------------------目录页------------------------%%
\newpage
\tableofcontents
% \thispagestyle{empty} % 目录不显示页码

%%------------------------正文页从这里开始-------------------%
\newpage
\setcounter{page}{1} % 让页码从正文开始编号

%%可选择这里也放一个标题
%\begin{center}
%    \title{ \Huge \textbf{{标题}}}
%\end{center}

\section{实验目的}


\begin{enumerate}[label=\roman*.]


\item \textbf{掌握STM32F103核心外设的应用}   - 通过实际操作**GPIO、ADC、定时器、中断、串口通信**等模块,深入理解单片机外设的工作原理及配置方法,培养对硬件资源的直接控制能力。

\item \textbf{培养嵌入式系统全流程开发能力}   - 从**硬件连接**(传感器、LED、按键)到**软件编程**(驱动开发、协议解析),完成完整的嵌入式系统设计流程,提升系统级工程思维。

\item \textbf{学习多模块协同与实时任务管理}    - 实现**ADC采集、PWM输出、串口通信、中断响应**等任务的协同工作,掌握基于中断和定时器的实时系统设计方法。

\item \textbf{提升调试与问题解决能力}    - 在硬件调试(如ADC噪声抑制、PWM波形验证)和软件调试(如中断冲突、内存溢出)中,积累嵌入式系统调试经验,掌握逻辑分析仪、串口调试工具的使用方法。

\item \textbf{培养工程文档编写与团队协作意识}    - 通过撰写实验报告、记录测试用例、维护代码注释,规范技术文档编写习惯;通过差异化任务分工,理解团队协作在复杂项目中的重要性。

\end{enumerate}

本实验旨在通过理论与实践结合,使学生掌握STM32F103单片机的综合开发技能,理解嵌入式系统的核心设计理念,并为后续复杂项目(如物联网终端、工业控制器)的开发奠定坚实基础。



\section{实验原理}

需要整个系统的实验原理,画原理框图,流程图,各个模块的原理。可以参考实验指导书和查阅相关的资料,可以有更能说明问题的图表。原则上字数不少于500字。

\begin{figure}[!htbp]
    \centering
    \includegraphics[width =.3\textwidth]{figures/hfutlogo.png}
    \caption{此处可以插入流程图}
    \label{Library}
\end{figure}


\section{实验内容}

需要整个系统的实验原理,画原理框图,流程图,各个模块的原理。可以参考实验指导书和查阅相关的资料,可以有更能说明问题的图表。原则上字数不少于500字。
\section{实验步骤}
\subsection{步骤1:插入高亮代码块}
利用\verb|lstlisting| 配置
\begin{lstlisting}[style=CPP, title="c++代码"]
#include <iostream>
#include <array>
int main()
{
    constexpr int MAX = 100;
    std::array<int, MAX> arr;
}  
\end{lstlisting}


这部分需要详细填写实验步骤。可以参考实验指导书。原则上字数不少于500
字。内容可写设计思路、设计过程,图文并茂为佳。





直接使用\verb|\cite{}|即可\cite{DBLP:conf/nips/VaswaniSPUJGKP17}。
\subsection{步骤2:文献引用}
例如:


   \textit{ 此处引用了文献}
   \cite{DBLP:conf/nips/VaswaniSPUJGKP17}。此处引用了文献\cite{DBLP:conf/nips/VaswaniSPUJGKP17}


引用过的文献会自动出现在参考文献中。

\section{实验结果与分析}
 (自行填写。每个实验项目的格式范例:
1) 关键流程分析
2) 实验结果
文字描述或者截图(所作的图)。必须有截图,截图数量不少于2幅。
3) 结果分析
对每一个结果,必须有相应分析,如解释图表反映的内涵、缘由,是否达到
预期目标,是否可改进等等。)
\section{总结及心得体会}

 (自行填写。必须写点什么,不能写“无”)
\section{对本实验过程及方法、手段的改进建议}


 (自行填写。必须写点什么,不能写“无”)
(注意:5,6部分能反映出实验的态度、方法和效果,应重点阐述,字数勿
少,独立完成,勿参考其他报告,避免雷同

\subsection{发布地址}
\begin{itemize}
    \item Github: \url{https://github.com/langsunny/HFUT_Report_LaTeX_Template}
\end{itemize}


%%----------- 参考文献 -------------------%%
%在reference.bib文件中填写参考文献,此处自动生成

\reference


\end{document}